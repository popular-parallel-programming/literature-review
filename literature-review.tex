\documentclass[a4paper]{article}

\usepackage{amsfonts}
\usepackage{amsmath}
\usepackage{color}
\usepackage{graphicx}
\usepackage{hyperref}
\usepackage[utf8]{inputenc}
\usepackage{listings}
\usepackage[square,sort,comma,numbers]{natbib}
\usepackage{todonotes}
\usepackage{url}

\lstset{
  breaklines=true,
  breakatwhitespace=true,
  basicstyle=\ttfamily,
}

\title{Systematic Mapping Study of Declarative Parallel Programming
  and Spreadsheet End-User Development}
\author{Florian Biermann\\\small{\texttt{fbie@itu.dk}}}

\begin{document}

\maketitle

\section{Introduction}
\label{sec:intro}

Spreadsheets are widely used in many different disciplines within
industry and science by domain experts who often are
non-programmers\todo{cite Sestoft 2014}. Research on spreadsheet
technology has recently gained increased interest, which is probably
due to a renewed popularity of functional programming
principles. Spreadsheets do not possess mutable state, which makes
reasoning about their correctness rather easy. Moreover, this
particular trait makes it possible to investigate transparent
parallelization of spreadsheet calculations.

In this systematic mapping study, we want to map out the publications
on spreadsheet end-user programming and declarative array programming
languages to inform further research on parallel programming in
spreadsheets. Declarative array programming allows a compiler to
re-write expressions to perform implicit parallelization of
computations. The user does not need to be aware of the parallelism
inherent to a problem and as long as they choose to use the highest
abstraction to model their computation, a sophisticated compiler can
infer such parallelism automatically.

With this study, we want to find\dots.\todo{Continue.}

\newpage{}
\part{A Systematic Mapping Study}
\label{part:syst-mapp-study}

\section{Protocol}
\label{sec:protocol}

\subsection{Background}
\label{sec:background}

The goal of this systematic mapping study is to coalesce two
apparently disjoint research areas, namely declarative parallel array
programming techniques and end-user development in a spreadsheet model
of computation.

This study is meant to motivate and inform further research on
declarative parallel programming in spreadsheets for end-users. We
want to gain an overview over well known techniques for transforming
expressions that describe seemingly sequential operations into
efficient parallel code. Furthermore, we want to gain insight into how
this can be combined with spreadsheet end-user development techniques
and if there has been made any effort into this direction
already. \todo{Should we maybe also include a search on spreadsheets
  AND parallel? It seems that there are only few publications on ACM
  and IEEE and then we have at least tried to find a connection.}

We purposely do not focus on data-flow parallelism in
spreadsheets. This is a separate project orthogonal to declarative
parallel array programming in spreadsheets.

To gain overlapping literature lists, we run three different search queries:

\begin{description}
\item[Declarative, parallel array programming languages] What are
  promising techniques for automatic parallelization of declarative,
  functional languages?
\item[Spreadsheet end-user development] Which paradigms have been
  developed, what do users use spreadsheets for?
\item[Parallelism in spreadsheets] To what degree have parallel
  spreadsheet engines been investigated? What are challenges and
  possibilities? Where does our agenda fit in?
\end{description}

\subsection{Research Questions}
\label{sec:research-questions}

Our research questions are listed in the following:

\begin{enumerate}
\item What is the state of the art in declarative parallel array
  programming languages?
\item What are the most scrutinized topics in array programming?
\item Has there been research on declarative parallel array
  programming in a spreadsheet model of computation?
\item Are there any obvious ways to combine declarative parallel array
  programming and spreadsheet end-user development?
\end{enumerate}

\subsection{Study Selection Criteria}
\label{sec:study-select-crit}

We perform two disjoint literature studies, one for declarative
parallel array programming languages and one for spreadsheet end-user
development.

In the following, we give a list of criteria for inclusion or
exclusion of studies. Naturally, some of the studies can fulfill
criteria of both lists. We choose therefore to perform a majority vote
on the number of fulfilled criteria when making a decision of
inclusion or exclusion.

\subsubsection{Criteria for Declarative Parallel Languages}
\label{sec:crit-decl-parall}

\paragraph{Inclusion Criteria}

A publication should be included if it:

\begin{itemize}
\item Mentions implementations of prominent array languages.
\item Focuses on automatic parallelization of array expressions.
\item Mentions caching and false sharing.
\item Talks about compiler optimizations.
\item Focuses on implementation of declarative, functional parallel
  programming techniques.
\item Mentions homogeneous systems and shared memory.
\item Mentions loop fusion and nested loops.
\item Talks generally about program transformation.
\end{itemize}

\paragraph{Exclusion Criteria}

A publication should be excluded if it:

\begin{itemize}
\item Focuses on the application of parallel programming, e.g.\ within
  machine learning.
\item Develops techniques for distributed memory or mentions message
  passing.
\item Develops techniques for focuses on formal verification.
\item Targets GPU, GPGPU, FPGA and hardware accelerated techniques on
  heterogeneous systems.
\item Includes I/O.
\item Works towards automatic parallelization of imperative languages.
\item Is a review paper without any novel contribution.
\item Focuses on SIMD and Accelerate. \todo{SIMD seems relevant? How did that get here? Cite Accelerate.}
\item Mentions storage, disk etc.
\item Focuses on transactional memory.
\end{itemize}

\subsubsection{Criteria for Spreadsheet End-User Development and
  Parallel Spreadsheets}
\label{sec:crit-spre-end}

\paragraph{Inclusion Criteria}

A publication should be included if it:

\begin{itemize}
\item Mentions functional programming or functional language.
\item Describes the implementation of a spreadsheet engine.
\item Focuses on gaining or providing spreadsheet understanding.
\item Mentions types or type-inference.
\item Is a survey of ``real-world spreadsheets'' and full spreadsheet corpora.
\end{itemize}

\paragraph{Exclusion Criteria}

A publication should be excluded if it:

\begin{itemize}
\item Focuses on the application of spreadsheets and specific
  spreadsheet models, such as simulations or in teaching.
\item Describes systems inspired by spreadsheets.
\item Mentions CSCW and knowledge work or performs ethnographic
  studies.
\item Focuses on data structures in spreadsheets.
\item Mentions either mashups, mobile apps or web development.
\item Focuses on external tools and architectures for spreadsheet
  users.
\item Develops techniques for transforming spreadsheets.
\end{itemize}

\subsection{Quality Assurance}
\label{sec:quality-assurance}

To verify that the results of our searches are meaningful, we compile
a list of relevant and important publications, which must be included
in the results obtained by our automated search\todo{Include
  Nlelloch's NESL, Peyton-Jones' User-Focused Approach to Functions
  and Sestoft's Spreadsheet Implementation Technology.}.

\section{Process}
\label{sec:process}

\subsection{Literature Search}
\label{sec:literature-search}

We have use IEEExplore and ACM Digital Library as sources for our
literature search. The number of results from these sources varies
drastically, which is probably due to IEEExplore interpreting search
queries very strictly. We avoid Google Scholar and CiteSeerX, as these
meta engines return way over a thousand publications for each query,
which is infeasible for our scope.

The uniqueness of a publication in the following means that a
publication is listed only once. Sometimes, publications have multiple
entries in a digital library, for instance one for a conference's
proceedings and one for SIGPLAN Notes.

We have an initial list of 681 publications to consider, which we
construct as described in the following:

\paragraph{Declarative Parallel Array Programming Languages}

To generate a literature list for declarative parallel array
programming languages, we use the following query:

\begin{lstlisting}
(functional AND array AND programming AND parallel) AND (data-parallel OR ``data parallel'' OR multi-core OR multicore  OR ``multi core'')
\end{lstlisting}

\noindent This results in 250 publications of which 194 are unique from the ACM
Digital Library. IEEExplore, however, only returns eight publications,
all of which are unique. This adds 202 publications to consider.

\paragraph{Spreadsheet End-User Development}

The search query we use for finding publications on spreadsheet
end-user development is:

\begin{lstlisting}
spreadsheets AND (end-user-development OR "end user development" OR "end-user development")
\end{lstlisting}

\noindent This results in 435 publications of which 386 are unique from the ACM
Digital Library and in 104 publications, all of which are unique, from
IEEExplore. This adds 424 publications to consider.

\paragraph{Parallel Spreadsheets}

We use the following query to generate a list of publications focusing
on anything parallel in spreadsheets:

\begin{lstlisting}
spreadsheets AND parallel
\end{lstlisting}

\noindent ACM returns 21 publications of which 17 are unique and
IEEExplore returns 38 publications. This adds 55 publications to
consider.

\subsection{Literature Selection}
\label{sec:literature-selection}

Using the criteria defined in Section~\ref{sec:study-select-crit}, we
include studies based on their titles and their abstracts. If the
title is not informative enough, we accept the study and will later
screen it again based on the abstract.

To perform the selection of literature move conveniently, we have
developed an Emacs-based tool, called the Systematic Literature Review
Mode (SLIRM). SLIRM automatically downloads abstracts and full-text
files on demand for BibTeX-formatted entries that have been exported
from a digital library, such as the ACM Digital Library. SLIRM is
open-source and freely available\footnote{Download SLIRM from
  \url{https://github.com/fbie/slirm}}.

\subsection{Mapping}
\label{sec:mapping}

\newpage{}
\part{Results}
\label{part:results}

\section{Declarative Parallel Programming}
\label{sec:declarative-parallel-programming}

\section{Spreadsheet End-User Development}
\label{sec:spreadsheet-end-user-dev}



\end{document}
