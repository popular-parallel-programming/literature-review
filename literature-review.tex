\documentclass[a4paper]{article}

\usepackage[square,sort,comma,numbers]{natbib}
\usepackage[utf8]{inputenc}
\usepackage{amsfonts}
\usepackage{amsmath}
\usepackage{color}
\usepackage{graphicx}
\usepackage{hyperref}
\usepackage{listings}
\usepackage{multicol}
\usepackage{todonotes}
\usepackage{url}

\lstset{
  breaklines=true,
  breakatwhitespace=true,
  basicstyle=\ttfamily,
}

\newcommand{\sac}{S\textsc{a}C}

\title{Declarative Parallel Programming in Spreadsheet End-User
  Development}

\author{Florian Biermann\\\small{\texttt{fbie@itu.dk}}}

\begin{document}

\maketitle

\section{Introduction}
\label{sec:intro}

Domain experts from widely different disciplines within industry and
science by domain use spreadsheets to model complex computations
conveniently~\cite{Sestoft2014Spreadsheet}. Research on spreadsheet
technology has recently gained increased interest, which is probably
due to a renewed popularity of functional programming
principles. Spreadsheets do not possess mutable state, which makes
reasoning about their correctness rather easy. Moreover, this
particular trait makes it possible to investigate transparent
parallelization of spreadsheet calculations. The widespread use of
spreadsheets together with the apparent opportunities for spreadsheet
parallelization provide a synergy for future research on ``popular
parallel programming''.

As an example of recent development in parallel spreadsheet
technology, AMD has implemented OpenCL kernel generation in
LibreOffice Calc. Their modification generates kernels from operations
on cell-area, which highly parallel GPGPU hardware then
executes~\cite{Trudeau2015Collaboration}.

In this literature review, we want to provide the reader with an
overview of the publications on spreadsheet end-user programming and
declarative array programming to inform further research on parallel
programming in spreadsheets with a focus on functional programming and
higher-order functions.

Declarative array programming allows the language run-time to re-write
expressions to perform implicit parallelization of computations. The
user does not need to be aware of the parallelism inherent to a
problem and as long as they choose to use the highest abstraction to
model their computation, a sophisticated compiler can infer such
parallelism automatically. We focus on approaches that do not require
special hardware, i.e.\ shared-memory multiprocessors.

\subsection{Approach}
\label{sec:approach}

Why and how we use tools from systematic literature reviews,
citing~\cite{keele2007guidelines,
  petersen2008systematic}. \todo{Write!}

We present an overview over the methodology in
Sections~\ref{sec:protocol} and~\ref{sec:process}.

\subsection{Research Questions}
\label{sec:research-questions}

We formulate our research questions as follows:

\begin{enumerate}
\item What is the state of the art in declarative parallel array
  programming languages?
\item What are the most scrutinized topics in array programming?
\item Has there been research on declarative parallel array
  programming in a spreadsheet model of computation?
\item Are there any obvious ways to combine declarative parallel array
  programming and spreadsheet end-user development?
\end{enumerate}

\subsection{Threats to Validity}
\label{sec:threats-validity}

What we did not do methodologically sound and what kinds of blind
spots there exist (that is, publications that are not listed in this
review). \todo{Write!}

\section{Declarative Parallel Programming}
\label{sec:declarative-parallel-programming}

We have identified four key topics within parallel array programming:
nested data parallelism, array fusion, data structures and loop
parallelization. While these clearly are interconnected and we
therefore cannot just see them all by themselves, we will use them to
structure the review of the research body on parallel array
programming.

There are a few key-languages that appear in the literature. The two
oldest are Fortran\todo{Source?}, which is an imperative
high-performance language, and APL~\cite{Iverson1962Programming},
which is a high-level declarative and functional
language. Furthermore, much research has focused on
Single-assignment-C (\sac{})\todo{Source?}, which is a C-like
functional language, and Data Parallel Haskell (DPH) and
REPA\todo{Source?}, both of which are library and compiler extensions
for Haskell. Another important language is
NESL~\cite{Blelloch1993NESL}, which introduced the idea of statically
flattening nested arrays.

The opportunity for parallelism in these languages stems from
bulk-operations over arrays. While imperative languages like Fortran
use iterative loops, functional languages depend on higher-order
functions like \texttt{map} or \texttt{fold} to express data-parallel
computations. \citet{Ching:1990:APA:97808.97826} argued that it is
easier for programmers to express parallelism in such a declarative
high-level way than by explicitly scheduling work to different
processors and implemented this approach in the APL370 compiler. The
only requirement to make use of such implicitly parallel constructs is
that the programmer writes idiomatic code which the language runtime
can parallelize
efficiently~\cite{Bernecky:2015:AEP:2774959.2774962}. Not only that
parallelism is much easier to extract, but also sequential programs
written in such a high-level style will often perform
faster~\cite{Bernecky:2015:AEP:2774959.2774962}.

\subsection{Nested Data Parallelism}
\label{sec:nest-data-parall}

Nested data parallelism describes the nesting of parallel operations
over a nested array. A standard example from the literature is matrix
multiplication~\cite{Keller:2010:RSP:1863543.1863582}. A matrix is
represented by an array of arrays. We can then implement matrix
multiplication conveniently via higher-order functions, here
illustrated in F\#:

\begin{lstlisting}[language=ML]
type Matrix = double [] [];
\end{lstlisting}
\todo{Actually write the code.}

Such a declarative implementation of matrix multiplication lends
itself naturally to an implicitly parallel execution. However, there
are possibilities for parallelism on different levels in the
three-dimensional array structure. Early work on the parallel language
Actulus~\cite{Perrott:1979:LAV:357073.357075} recognizes the
difficulty of nested data parallelism and as a temporary solution
restricts the level of parallelism to one dimension which the
programmer has to choose ahead of time.

A naive way of implementing such nested parallelism would be to simply
start new parallel threads from within each parallel thread. There
would be quite some overhead to this solution. Research on nested data
parallelism has focused on eliminating this overhead in two different
ways.

\paragraph{Flattening}

\citet{Blelloch1993Implementation} introduced the idea of statically
flattening nested arrays to perform an optimal parallelization of
operations over arrays. The key to doing so is the right choice of
array representation and there exists a representation which enables
the compiler to flatten arrays in constant time. NESL implements this
flattening scheme and \citet{Blelloch:1996:PTS:232627.232650} showed,
based on NESL's built-in cost semantics, that NESL actually can be
implemented without any additional overhead due to flattening.

Other researchers have since taken up the idea of flattening nested
parallelism again by \citet{Lippmeier:2012:WEH:2364527.2364564} and
implemented it in Haskell. In contrast to NESL, Haskell is a
fully-fledged functional programming language with types and
higher-order functions that is compiled instead of interpreted.

As it turns out, flattening nested data parallelism is much harder to
achieve in such a higher-order language. The presence of higher-order
functions makes lifting of functions more
complicated~\cite{Lippmeier:2012:WEH:2364527.2364564}. Higher-order
flattening introduces many intermediate arrays that take time and
space. Therefore, \citet{Keller:2012:VA:2364506.2364512} developed a
technique to avoid such flattening in such cases.

The more recent high-level language NOVA builds on the principles
developed in NESL and DPH, using a lot of the state-of-the-art
techniques and targets both, shared-memory multiprocessors as well as
general purpose GPUs~\cite{Collins:2014:NFL:2627373.2627375}.

Flattening nested data parallelism is tightly coupled to array fusion,
which we will discuss in detail in Section~\ref{sec:fusion}.

\paragraph{Dynamic scheduling}

Static flattening nested data parallelism without overhead targets
homogeneous computations on possibly irregular arrays. Possible
imbalance of work distribution is a result of irregular nested data
structures, such as sparse vectors, where adjacent zero-valued entries
are not stored in memory. This demands a strong type system, such that
all elements of an array are of the same type. Otherwise, a
two-dimensional array might mostly contain numbers but spuriously also
other data types or even reference arrays. This means that not the
entire parallelism is visible to the compiler, so static flattening
might not be feasible to balance the workload.

Dynamic scheduling leaves the distribution of work to run-time of the
program. \sac{}, for instance, uses work-stealing
queues~\cite{Chase2005Dynamic, Grelck:2007:SOS:1248648.1248654} to
dynamically schedule work. Such scheduling schemes can produce some
overhead at run-time that we otherwise could alleviate by using
compile-time transformations. To minimize overhead for different kinds
of computations, \citet{Fluet:2008:SFG:1411204.1411239} argue for a
mix of schedulers and to let the run-time chose which scheduler to use
for distributing work.

A notable scheduling heuristic is lazy tree
splitting~\cite{Bergstrom:2010:LTS:1863543.1863558}. In this scheme,
every thread gets assigned some part of the workload. Threads
communicate with other threads via a technique inspired by and as
efficient as work-stealing queues. Representing arrays via balanced
binary trees makes halving arrays a constant time operation. When a
worker thread iterates over an array, it checks at every n-th
iteration step whether any of the remaining threads are idle. If so,
it splits its remaining array in half, dispatches the latter half to
the idle threads and continues to the next iteration step. This
heuristic shows low overhead in experimental
settings~\cite{Bergstrom:2010:LTS:1863543.1863558}.

\subsection{Fusion}
\label{sec:fusion}

Fusion refers to avoiding intermediate representations of arrays for
consecutive bulk operations. For instance, we can fuse two succeeding
applications of \texttt{map} as follows, where \texttt{.} is the
function composition operator:

\begin{center}
  \begin{tabular}{ccc}
\begin{lstlisting}[language=ML]
map g (map f xs)
\end{lstlisting}
    & $\Longrightarrow$ &
\begin{lstlisting}[language=ML]
map (f . g) xs
\end{lstlisting}
  \end{tabular}
\end{center}

This optimization is very valuable in all kinds of programs,
sequential and parallel alike, for example in sequential
Fortran~90 programs\cite{Hwang:1995:AOS:209936.209949}.

Fusion is a static technique performed during compile
time. \citet{Chakravarty:2001:FAF:507635.507661} express the fusion
transformations as straight-forward equational rewrite rules. They
also observe that flattening nested arrays makes fusion a much simpler
task which emphasizes the connection between flattening and
fusion. Some researchers have been focusing on making such
optimizations visible to the programmer via
types~\cite{Lippmeier:2012:GPA:2364506.2364511}. This enables
programmers to reason about the performance of their declarative code.

More aggressive fusion is possible if the compiler performs a more
complex analysis. \citet{Henriksen:2013:TGA:2502323.2502328} use a
data-flow based graph-reduction to analyze functional programs with
second-order functions on arrays for fusion possibilities. Their
analysis can detect code structures that inhibit fusion, subsequently
re-writes the program in such a way that fusion becomes possible and
avoids duplicate computations

A special variant of fusion is destructive update analysis. Since data
structures are immutable in purely functional languages, writing to a
single index of an array produces in a new array. The update operation
copies all elements from the original array with exception of the
index that it updated. If the updated array is not referenced again
throughout the program, we can perform the update in-place, or
destructively, without the overhead of copying all
data. \citet{Sastry:1994:PDU:182409.182486} developed an analysis for
destructive array updates using live-variable data-flow analysis.

\subsection{Data Structures}
\label{sec:data-structures}

Choosing the right data structure is a key element to high-performance
array computing. \citet{Lowney:1981:CAI:567532.567533} developed
carrier arrays as an extension to APL that are able to express
irregular nested parallelism, opposite to regular nested parallelism
where all sub-arrays must be of the same length. Carrier arrays decide
automatically to which rank a function must be lifted to, to be
applied to all elements of the array.

Performance of single operations on arrays are also of concern. For
instance, functional arrays should be able to perform constant-time
lookup and preferably also constant-time update~\cite{47507}. It is
well-known that data structures choose a trade-off between the
asymptotic performance of the different operations, like random
access, update, appending etc. Nevertheless,
\citet{Stucki:2015:RVP:2784731.2784739} developed a general-purpose
array data structure, the relaxed-radix-bound (RRB) array. They proved
the bounds for all operations and for practical sizes of an RRB array
to be constant or amortized constant. RRB arrays are of clear value to
parallel array programming, especially in conjunction with scheduling
schemes.

It is a challenge to implement such high-performance arrays in a
purely functional language. \citet{Arvind:1989:IDS:69558.69562}
developed I-Structures which conceptually are write-once arrays. Each
subscript can be written exactly once during the entire program. Reads
and writes to indices can be re-ordered according to re-write rules at
the cost of sacrificing referential transparency.

Type-based run-time specialization of functions and optimization of
data structures seems to be a widely applicable technique in
functional
programming~\cite{Hall:1994:UHT:182409.156781}. REPA~\cite{Keller:2010:RSP:1863543.1863582}
uses types in order to represent the (irregular) shapes of arrays and
to specialize higher-order functions. REPA arrays consist of unboxed
values and are lazy. Forcing a single element of an array forces the
evaluation of the entire array. Arrays can evaluate in parallel. Lazy
arrays can avoid creating intermediate arrays and therefore do not
require fusion.

\subsection{Loop Parallelization}
\label{sec:loop-parallelization}

Parallelism in imperative languages is expressed via some kind of
\texttt{do-loop} construct that roughly translates to ``for each
element of a list or for each integer in some range, perform the given
body''. The problem with this style of parallelism is that imperative
languages allow for non-trivial data dependencies and side-effects. As
a consequence, the analysis that a compiler needs to perform in order
to safely parallelize a \texttt{do-loop} is much more complicated than
when using higher-order functions.

One way to safely parallelize imperative loops is to enforce a
read-write order that can be computed at compile
time~\cite{Tang:1990:CTD:77726.255155}. When the read-write order of
elements is known, a reading thread synchronizes with the writing
thread to avoid lost updates. A similar approach uses a data-flow
analysis of accesses to array-subscripts on an inter-procedural
level~\cite{Maydan:1993:AFA:158511.158515}. Instead of enforcing an
ordering of read and write accesses,
\citet{Knobe:1998:ASF:268946.268956} simply fall back to
single-assignment arrays, thereby eliminating a whole class of data
dependencies in loops. The most dominant technique for data dependency
analysis is data-flow analysis of loops and array
indices~\cite{Maydan:1993:AFA:158511.158515,
  Knobe:1998:ASF:268946.268956}.

In bounded, iterative loops, the bounds must be checked after every
iteration. To alleviate this costly and repetitive computation,
\citet{Henriksen:2014:BCI:2627373.2627388} lift the bounds check out
of the loop at compile time and specialize the body for the bounds of
the loop.

These techniques for imperative loop-parallelization are relevant for
the implementation of high-level operations over arrays.

\subsection{Other Topics}
\label{sec:other-topics}

This section summarizes research that does not fit into the four major
categories.

\paragraph{Domain-specific languages}

Some of the research on array programming focused on domain-specific
languages (DSLs). \citet{4228136} implemented explicitly coordination
of parallel computations in \sac{} via an embedded DSL in a
declarative fashion. This, however, removes some of the declarative
nature of program code. Another DSL on top of \sac{} is Staged\sac{}
which adds compile-time shape inference of nested arrays to
\sac{}~\cite{Ureche:2012:SCS:2103746.2103762}. The compiler adds all
statically unsatisfiable requirements as run-time checks to the
program.

\paragraph{Retran}

Retran is a declarative, Fortran-like, purely functional
language~\cite{367042}. Similar to the APL extension by
\citet{Lowney:1981:CAI:567532.567533}, Retran automatically applies
lower-rank functions to all elements of a higher-rank
array. Therefore, there are no higher-order functions in
Retran. Retran uses anti-currying to lift functions to the required
rank.

\paragraph{Remap and data layout}

Declarative functional programming languages tend to free the
programmer from thinking explicitly about data layout. For performance
reasons, it can be necessary to expose data layout to the
programmer. Also, high-level languages often provide functions to
transpose two-dimensional arrays or to more generally change the
layout of an array. The most general operator of this kind is
\texttt{remap}, sometimes also referred to as \texttt{scatter} on the
assignment's right-hand side and \texttt{gather} on the assignment's
left-hand side.

\citet{Walinsky:1990:FPL:91556.91610} implemented implicit remapping
at compile time for functional programming languages. They provide
inference rules for remapping. By using such a remapping, they
effectively avoid intermediate representations of arrays. Implicit
remapping also aids improving vectorization of higher-order functions,
as shown by \citet{Sinkarovs:2013:SDL:2502323.2502332}.

The ZPL programming language is based around regular-shaped arrays and
a concept called regions which roughly equal named
index-sets~\cite{Chamberlain1999Regions}. Furthermore, it makes all
communication visible to the programmer through syntax and types,
featuring a ``what you see is what you get'' performance
model~\cite{Chamberlain1998ZPLs}. Data remapping is a crucial
operation in ZPL.\@ \citet{Deitz:2003:DIP:781498.781526} show ZPL's
\texttt{scatter}-\texttt{gather} operator semantics. The operator can
modify data layout of arbitrarily ranked arrays and exhibits high
performance.

\section{Spreadsheet Technology}
\label{sec:spreadsheet-end-user-dev}

Spreadsheets are visual programming environments. A collection of
spreadsheets is a workbook. A spreadsheet contains a rectangular grid
of cells. Each cell contains either a constant or a formula that
computes a value and possibly referencing other cells.

One major topic in research on spreadsheet programming is the lack of
abstraction: spreadsheets bundle data and logic in a single
representation~\cite{Isakowitz:1995:TLT:195705.195708}. Moreover,
spreadsheets encourage copying of formulas across cells to replicate
computations~\cite{1173080, Benfield:2009:FFD:1668113.1668121}. This
lack of abstraction makes spreadsheets less powerful than general
purpose programming languages~\cite{Miller:2015:SPB:2814189.2814201}.

Another main topic is general programming paradigms in a spreadsheet
model of computation. Researchers have augmented spreadsheets with
object orientation~\cite{Benfield:2009:FFD:1668113.1668121} and more
declarative programming
approaches~\cite{Stadelmann:1993:SBC:168642.168664,
  Singh:2016:TSD:2837614.2837668} which we will look at in greater
detail in Section~\ref{sec:progr-parad}. Again, even though we cannot
strictly separate abstraction and programming paradigms, this
categorization is convenient for the discussion of how researchers
propose to handle the complexity of spreadsheet models.

There are many more topics in spreadsheet research, i.e.\ testing of
spreadsheets, visualization or analysis of ``real-life'' spreadsheet
corpora, to only name a few. We focus however on research on end-user
programming in a spreadsheet model of computations and will therefore
not consider these other topics here.

\subsection{Abstraction}
\label{sec:abstraction}

It is convenient to define two groups of spreadsheet abstraction. That
is (1) manual abstraction, where the user has the means to build their
own abstractions so to hide implementation details and (2) automatic
abstraction, where the user constructs spreadsheets in a familiar way
and the system later analyzes them to infer the logic and to
(subsequently) separate it from data.

\paragraph{Manual abstraction}

Many researchers observed that spreadsheets lack the most basic
abstraction of general-purpose programming languages: named
functions~\cite{Jones:2003:UAF:944705.944721}. Named functions make it
possible to hide implementation detail that is not important for the
overall logic of a specific model. Therefore,
\citet{Jones:2003:UAF:944705.944721} proposed to allow end-users to
define their own abstractions in terms of spreadsheet
computations. Each newly introduced function is essentially a
spreadsheet prototype that has designated input cells and a designated
output cell. Each time the user calls such a sheet-defined function, a
new spreadsheet instance of this function is instantiated to perform
the computation.

\citet{Sestoft:2008:IFS:1370847.1370867} extended upon this idea by
allowing sheet-defined, recursive and run-time compiled
functions. This approach is more general and alleviates the need for
instantiating explicit spreadsheets. This approach is implemented in
the experimental spreadsheet engine Funcalc which is described in
great detail in~\cite{Sestoft2014Spreadsheet}.

\paragraph{Automatic abstraction}

Researchers have developed systems that analyze spreadsheets to infer
their logic and to (subsequently) separate logic from data. This
allows users to build spreadsheets in a familiar
manner. \citet{Isakowitz:1995:TLT:195705.195708} developed a system to
automatically performs such a separation and manages spreadsheet logic
for modular re-use. They observe that the majority of spreadsheet
errors they encounter not are simple off-by-one reference errors and
typos but severe errors in the model's logic which they relate to
classic programming errors where the programmer has not chosen a high
enough level of abstraction.

The visual layout of spreadsheets is often implicit documentation of
the logic. \citet{1173080} however, observe that this visual layout
often leads to misconceptions if another user takes over the
spreadsheet. Therefore, they developed a set of logical and semantic
equivalence classes for cells. These equivalence classes help
visualizing repetitions in spreadsheet grids, which are the high-level
structures a user needs to understand in order to be able to maintain
the spread sheet.

Types are also useful abstractions over spreadsheets. The literature
includes different type inference systems that make the user aware of
formulas where the expected type differs from the actual
type~\cite{Abraham:2006:TIS:1140335.1140346,
  Cheng2015Static}. Researchers have proposed different solutions to
handle types in spreadsheets and a common problem is efficient typing
of cell areas~\cite{Abraham:2006:TIS:1140335.1140346,
  Cheng2012Abstract}. Type systems are part of automatic abstraction
because the types are inferred rather than annotated.

\subsection{Programming Paradigms}
\label{sec:progr-parad}

Researchers have also proposed to apply different programming
paradigms to the spreadsheet domain with a focus on raising the
abstraction level.

\paragraph{Object orientation}

Functional Model Development
(FMD)~\cite{Benfield:2009:FFD:1668113.1668121} is a domain-specific
language for Excel and exposes objects to spreadsheet users. Objects
are an accumulation of data with some functions defined on these
objects. Users can use a special syntax to declare variables that
model input parameters for user-defined functions. Functions are
defined inline (as in on the same spreadsheet) by prototype formulas
where the cell that would yield the result actually evaluates to the
new defined function. As spreadsheets encourage copying of formulas
over the same row or column, FMD introduces a high-level \texttt{map}
operator that applies the same user-defined function across a column
or row.

To apply stronger separation of implementation and instantiation,
\citet{6070409} developed ClassSheet. The logic of a computation is
defined in a model-spreadsheet, while each common spreadsheet that
computes the model is an instance of this model-spreadsheet. This
approach resembles a manual version of the abstraction model by
\citet{Isakowitz:1995:TLT:195705.195708}.

\paragraph{Constraint programming}

\citet{Stadelmann:1993:SBC:168642.168664} proposed to let spreadsheet
users express their models by means of providing the system with
constraints to solve. This approach severely reduces the amount of
code required to model complicated
logic~\cite{Stadelmann:1993:SBC:168642.168664}. However, due to the
requirement of being able to name a cell multiple times in a
constraint, it is infeasible to let constraints directly replace cell
formulas. Instead, the system provides a second window that contains
constraints. This side-steps the spreadsheet model slightly.

\paragraph{By example}

Programming by example allows users to explain how to transform data
by performing a few transformations manually, from which the system
can infer general transformation rules. This is useful for
bulk-processing similar items. \citet{Singh:2016:TSD:2837614.2837668}
extended Excel with a DSL that allows users to provide such example
transformations such that Excel then automatically transforms the
remaining items. They combine a probabilistic approach of parsing with
joint learning of transformation rules. They require however that the
data type that a user wants to transform already exists as a
predefined model.

\section{Discussion}
\label{sec:discussion}

Much research on spreadsheet end-user programming focuses on bulk-data
transformations. Also, researchers aim to avoid repetition in
spreadsheets to avoid error sources. Semantic spreadsheet analysis
reveals repetitive patterns of homogeneous computations that remind of
explicit mappings of functions.

These observations fit nicely to declarative, higher-order array
programming. \todo{Extend!}

\section*{Acknowledgements}

\todo{Acknowledge!}

\bibliographystyle{unsrtnat}
\bibliography{array-programming-acm-accepted,array-programming-ieee-accepted,spreadsheets-accepted-acm,spreadsheets-ieee-accepted,spreadsheets-parallel-accepted,fbie}

\appendix

\newpage{}

\section{Protocol}
\label{sec:protocol}

\subsection{Background}
\label{sec:background}

The goal of this systematic mapping study is to coalesce two
apparently disjoint research areas, namely declarative parallel array
programming techniques and end-user development in a spreadsheet model
of computation. We base our methodology on~\citet{keele2007guidelines}
and~\citet{petersen2008systematic}.

This study is meant to motivate and inform further research on
declarative parallel programming in spreadsheets for end-users. We
want to gain an overview over well known techniques for transforming
expressions that describe seemingly sequential operations into
efficient parallel code. Furthermore, we want to gain insight into how
this can be combined with spreadsheet end-user development techniques
and if there has been made any effort into this direction
already. \todo{Should we maybe also include a search on spreadsheets
  AND parallel? It seems that there are only few publications on ACM
  and IEEE and then we have at least tried to find a connection.}

We purposely do not focus on data-flow parallelism in
spreadsheets. This is a separate project orthogonal to declarative
parallel array programming in spreadsheets.

To gain overlapping literature lists, we run three different search queries:

\begin{description}
\item[Declarative, parallel array programming languages] What are
  promising techniques for automatic parallelization of declarative,
  functional languages?
\item[Spreadsheet end-user development] Which paradigms have been
  developed, what do users use spreadsheets for?
\item[Parallelism in spreadsheets] To what degree have parallel
  spreadsheet engines been investigated? What are challenges and
  possibilities? Where does our agenda fit in?
\end{description}

\subsection{Study Selection Criteria}
\label{sec:study-select-crit}

We perform two disjoint literature studies, one for declarative
parallel array programming languages and one for spreadsheet end-user
development.

In the following, we give a list of criteria for inclusion or
exclusion of studies. Naturally, some of the studies can fulfill
criteria of both lists. We choose therefore to perform a majority vote
on the number of fulfilled criteria when making a decision of
inclusion or exclusion. The criteria lists can be seen as disjunctions
of the single criteria.

\paragraph{Declarative Parallel Array Programming}
~\\

We include a publication if it:

\begin{itemize}
\item Mentions implementations of prominent array languages.
\item Focuses on automatic parallelization of array expressions.
\item Mentions caching and false sharing.
\item Talks about compiler optimizations.
\item Focuses on implementation of declarative, functional parallel
  programming techniques.
\item Mentions homogeneous systems and shared memory.
\item Mentions loop fusion and nested loops.
\item Talks generally about program transformation.
\end{itemize}

We exclude a publication if it:

\begin{itemize}
\item Focuses on the application of parallel programming, e.g.\ within
  machine learning.
\item Develops techniques for distributed memory or mentions message
  passing.
\item Develops techniques for focuses on formal verification.
\item Targets GPU, GPGPU, FPGA and hardware accelerated techniques on
  heterogeneous systems.
\item Includes I/O.
\item Works towards automatic parallelization of imperative languages.
\item Is a review paper without any novel contribution.
\item Mentions Accelerate.\todo{Cite Accelerate.}
\item Mentions storage, disk etc.
\item Focuses on transactional memory.
\end{itemize}

\paragraph{Spreadsheet End-User Development and Parallel Spreadsheets}
~\\

We include a publication if it:

\begin{itemize}
\item Mentions functional programming or functional language.
\item Describes the implementation of a spreadsheet engine.
\item Focuses on gaining or providing spreadsheet understanding.
\item Mentions types or type-inference.
\end{itemize}

We exclude a publication if it:

\begin{itemize}
\item Focuses on the application of spreadsheets and specific
  spreadsheet models, such as simulations or in teaching.
\item Describes systems inspired by spreadsheets.
\item Mentions CSCW and knowledge work or performs ethnographic
  studies.
\item Focuses on data structures in spreadsheets.
\item Mentions either mashups, mobile apps or web development.
\item Focuses on external tools and architectures for spreadsheet
  users.
\item Develops techniques for transforming spreadsheets.
\item Surveys ``real-life'' spreadsheets.
\end{itemize}

\subsection{Quality Assurance}
\label{sec:quality-assurance}

To verify that the results of our searches are meaningful, we compile
a list of relevant and important publications, which must be included
in the results obtained by our automated search\todo{Include
  Nlelloch's NESL, Peyton-Jones' User-Focused Approach to Functions
  and Sestoft's Spreadsheet Implementation Technology.}.

\section{Process}
\label{sec:process}

\subsection{Literature Search}
\label{sec:literature-search}

We have use IEEExplore and ACM Digital Library as sources for our
literature search. The number of results from these sources varies
drastically, which is probably due to IEEExplore interpreting search
queries very strictly. We avoid Google Scholar and CiteSeerX, as these
meta engines return way over a thousand publications for each query,
which is infeasible for our scope.

The uniqueness of a publication in the following means that a
publication is listed only once. Sometimes, publications have multiple
entries in a digital library, for instance one for a conference's
proceedings and one for SIGPLAN Notes.

We have an initial list of 681 publications to consider, which we
construct as described in the following:

\paragraph{Declarative Parallel Array Programming}

To generate a literature list for declarative parallel array
programming languages, we use the following query:

\begin{lstlisting}
(functional AND array AND programming AND parallel) AND (data-parallel OR ``data parallel'' OR multi-core OR multicore  OR ``multi core'')
\end{lstlisting}

\noindent This results in 250 publications of which 194 are unique from the ACM
Digital Library. IEEExplore, however, only returns eight publications,
all of which are unique. This adds 202 publications to consider.

\paragraph{Spreadsheet End-User Development}

The search query we use for finding publications on spreadsheet
end-user development is:

\begin{lstlisting}
spreadsheets AND (end-user-development OR "end user development" OR "end-user development")
\end{lstlisting}

\noindent This results in 435 publications of which 386 are unique from the ACM
Digital Library and in 104 publications, all of which are unique, from
IEEExplore. This adds 424 publications to consider.

\paragraph{Parallel Spreadsheets}

We use the following query to generate a list of publications focusing
on anything parallel in spreadsheets:

\begin{lstlisting}
spreadsheets AND parallel
\end{lstlisting}

\noindent ACM returns 21 publications of which 17 are unique and
IEEExplore returns 38 publications. This adds 55 publications to
consider.

\subsection{Literature Selection}
\label{sec:literature-selection}

Using the criteria defined in Section~\ref{sec:study-select-crit}, we
include studies based on their titles and their abstracts. If the
title is not informative enough, we accept the study and will later
screen it again based on the abstract.

To perform the selection of literature move conveniently, we have
developed an
Emacs\footnote{\url{https://www.gnu.org/software/emacs/}}-based tool,
called the Systematic Literature Review Mode (SLIRM). SLIRM
automatically downloads abstracts and full-text files on demand for
BibTeX-formatted entries that have been exported from a digital
library, such as the ACM Digital Library. SLIRM is open-source and
freely available\footnote{Download SLIRM from
  \url{https://github.com/fbie/slirm}}.

\end{document}
